\documentclass{scrartcl}\usepackage[utf8]{inputenc}
\usepackage[T1]{fontenc}
\usepackage{lmodern}
\usepackage[ngerman]{babel}
\usepackage{amsmath}
\usepackage{amssymb}
\usepackage{stmaryrd}
\usepackage{blindtext}
\title{6. Binomialkoeffizienten}
\author{Adrian Hille}
\begin{document}
\Large 6. Binomialkoeffizienten \\
\\
\normalsize
$\binom{n}{k}$ (gesprochen: n \"uber k)\\
die Anzahl der k-elementigen Teilmengen einer n-elementigen Menge.\\
Bem.: \\
$\binom{n}{0} =1$\\
$\binom{n}{n}=1$\\ 
...sind Extremf\"alle\\
\\
Schreiben n! f\"ur $1 \cdot 2 \cdot ... \cdot n$, definieren $0! := 1$\\
\\
\textbf{Proposition}\\
F\"ur alle nat\"urlichen Zahlen n, k gilt: 
$\binom{n}{k} = \frac{n!}{k!(n-k)!}$\\
\\
Insbesondere gilt:\\
$\binom{n}{2} = \frac{n \cdot (n-1) }{2}$\\
\\
Wollen zeigen: $\binom{n}{k} = \frac{n!}{k!(n-k)!}$\\
Bilden k-elementige Teilmenge einer n-elementigen Menge: w\"ahle erstes Element, dann ein zweites usw., bis zum k-ten Element.\\
Daf\"ur gilt: $n \cdot (n-1) \cdot (...) \cdot (n-k+1) = \frac {n!}{(n-k!)}$ M\"oglichkeiten.\\
Auswahlreihenfolge spielt keine Rolle:\\
jeweils k! M\"oglichkeiten f\"uhren zur gleichen Teilmenge. $\square $\\
\\
Beobachtung: F\"ur alle nat\"urlichen Zahlen n,k gilt:\\
$\binom{n}{k}={n}{n \cdot k}$\\
(kombinatorischer Beweis:) Doppeltes Abz\"ahlen:\\
  \begin{enumerate}
    	\item Aus n Spielern einer Mannschaft mit k Spielern aufstellen.
    	\item Aus n Spielern n-k Spieler ausw\"ahlen, die nicht spielen.
    \end{enumerate}
(algebraischer Beweis:)\\
Folgt direkt aus $\binom{n}{k} = n \binom{n!}{k!(n-k)!} \qquad \square $\\
\\
Beobachtung: F\"ur alle nat\"urlichen Zahlen n,k gilt:\\
$k \binom{n}{k} = n \binom{n-1}{k-1}$\\
(kombinatorischer Beweis): Doppeltes Abz\"ahlen\\
  \begin{enumerate}
    	\item Aus n Spielern einer Mannschaft mit k Spielern inklusive Kapit\"an aufstellen.
    	\item Aus n Spielern einen Kapit\"an ausw\"ahlen und dann aus den \"ubrigen k-1 Spieler ausw\"ahlen
    \end{enumerate}
(algebraischer Beweis):\\
$k \binom{n}{k} = k \frac{n!}{k!(n-k)!} = \frac{n!}{(k-1)!(n-k)!} = n\frac{(n-1)!}{(k-1)!((n-1)-(k-1))!} = n \binom{n-1}{k-1} \qquad \square $ \\
\\
Beobachtung: F\"ur alle nat\"urlichen Zahlen n, k gilt: \\
\[ \sum_{k=0}^n =\binom{n}{k} = 2^n \]\\
Es gilt $2^n$ Teilmengen einer n-elementigen Menge.\\
\\
\textbf{Pascalsches Dreieck}\\
\begin{tabular}{rccccccccc}
$n=0$:&    &    &    &    &  1\\\noalign{\smallskip\smallskip}
$n=1$:&    &    &    &  1 &    &  1\\\noalign{\smallskip\smallskip}
$n=2$:&    &    &  1 &    &  2 &    &  1\\\noalign{\smallskip\smallskip}
$n=3$:&    &  1 &    &  3 &    &  3 &    &  1\\\noalign{\smallskip\smallskip}
$n=4$:&  1 &    &  4 &    &  6 &    &  4 &    &  1\\\noalign{\smallskip\smallskip}
\end{tabular}
\\
F\"ur alle nat\"urlichen Zahlen n, k gilt: \\
$\binom{n+1}{k}= \binom{n}{k-1}+\binom{n}{k}$\\
\\
Beweis: Sei M eine (n+1) elementige Menge und $x \in M $\\
w\"ahlen x aus $\binom{n}{k-1}$ M\"oglichkeiten.\\
w\"ahlen x nicht aus $\binom{n}{k}$ M\"oglichkeiten. \qquad $\square$\\
\end{document}
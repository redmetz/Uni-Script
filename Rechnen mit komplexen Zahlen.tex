\documentclass{scrartcl}
\usepackage[utf8]{inputenc}
\usepackage[T1]{fontenc}
\usepackage{lmodern}
\usepackage[ngerman]{babel}
\usepackage{amsmath}
\usepackage{amssymb}
\title{Lineare Algebra - Rechnen mit den komplexen Zahlen}
\author{Tobias Raaf}
\begin{document}
	\Large Rechnen mit $\mathbb{C}$, wie in $\mathbb{R}$ mit \(i^{2}=-1\)\\
	\normalsize\\
	Es sei $z_{1}=a+bi, z_{2}=c+di$\\
	\textbf{Addition}: (a+bi)+(c+di)=(a+c)+(b+d)*i\\
	\textbf{Substraktion}: (a+bi)-(c+di)=(a-c)+(b-d)*i\\
	\textbf{Multiplikation}: (a+bi)*(c+di)=ac+adi+bci+bdi\(^{2}\)=(ac-bd)+(ad+cb)*i\\
	\textbf{Division}: $\dfrac{a+bi}{c+di}$*1=$\dfrac{a+bi}{c+di}$*$\dfrac{c-di}{c-di}$, mit c+di$\neq$0\\
	Definition: Für z=a+bi $\in \mathbb{C}$ nennt man $\bar{z}$=a-bi, die zu z konjugiert komplexe Zahl.\\
	\\
	\textbf{Bemerkung: ($\mathbb{C}$;+,*) nennt man Körper der komplexen Zahlen}
\end{document}
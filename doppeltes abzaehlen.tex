\documentclass{scrartcl}\usepackage[utf8]{inputenc}
\usepackage[T1]{fontenc}
\usepackage{lmodern}
\usepackage[ngerman]{babel}
\usepackage{amsmath}
\usepackage{amssymb}
\usepackage{stmaryrd}
\usepackage{blindtext}
\title{5. Doppeltes Abz\"ahlen}
\author{Adrian Hille}
\begin{document}
\Large 5. Doppeltes Abz\"ahlen\\
\\
\normalsize
Das doppelte Abz\"ahlen ist ein kombinatorisches Beweisprinzip.\\
\\
Z\"ahle auf zweierlei Weise die Kardinalit\"at z von:\\
$\{(t_i , t_j) \mid t_i~gibt~ t_j~ die~ Hand\}$\\
$x_i$: Anzahl der Personen denen $t_i$ die Hand reicht\\
y: Anzahl aller Handschl\"age\\
Es gilt: \[ \sum_{i=1}^n = x_i=z=2y \]\\
und es folgt, dass eine gerade Anzahl der $x_i$ ungerade sein muss. $\square$ \\
\end{document}
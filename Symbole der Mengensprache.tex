\documentclass{scrartcl}\usepackage[utf8]{inputenc}
\usepackage[T1]{fontenc}
\usepackage{lmodern}
\usepackage[ngerman]{babel}
\usepackage{amsmath}
\usepackage{amssymb}
\usepackage{stmaryrd}
\usepackage{blindtext}
\title{1- DIS- Die Symbole der Mengensprache}
\author{Adrian Hille}
\begin{document}
\Large 1. Die Symbole der Mengensprache
\\
\\
\normalsize
Mengen: \\ 
...jede Zusammenfassung M von bestimmten wohlunterschiedenen Objekten in unserer Anschauung oder des Denkens.\\
\\
Schreibweisen: 
   \begin{itemize}
    	\item e $\in$ M: e ist Element der Menge M.
    	\item e $\notin$ M: e ist kein Element der Menge M.
  	\item $\varnothing$: leere Menge.
    	\item A $\subseteq$ B: A ist Teilmenge von B.
    \end{itemize}
\end{document}
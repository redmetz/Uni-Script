\documentclass{scrartcl}
\usepackage[utf8]{inputenc}
\usepackage[T1]{fontenc}
\usepackage{lmodern}
\usepackage[ngerman]{babel}
\usepackage{amsmath}
\usepackage{amssymb}
\title{Lineare Algebra - Der Körper der komplexen Zahlen}
\author{Tobias Raaf}
\begin{document}
\Large 1. Der Körper der komplexen Zahlen - $\mathbb{C}$
\\
\\
\\
\normalsize
Ziel: Das Ziehen von Wurzeln aus negativen Zahlen.
\\
\\
\\
Bedingungen: $\mathbb{R}\subseteq\mathbb{C}$, \textit{i}$\in\mathbb{C}$, $\mathbb{C}$ soll so klein wie möglich sein und die grundlegenden Rechengesetze der Reellen Zahlen sollen gelten (+,-,*,:)\\
\\
Definitionen: \textit{i} mit \textit{\(i^{2}\)}=-1 nennt die imaginäre Einheit\\
$\mathbb{C}=\{a+bi|a,b\in\mathbb{R}\}$, heißt Menge der komplexen Zahlen\\
Es gilt: \textit{z}=a+bi\\
Bemerkung: Da $a\in\mathbb{R}\subseteq\mathbb{C}$, $b\in\mathbb{R}\subseteq\mathbb{C}$ und $i\in\mathbb{C},$\\ z=a+bi$\in\mathbb{C}$.\\
Definiere: Den Realteil Re(z)=a und den Imaginärteil Im(z)=b
	
\end{document}
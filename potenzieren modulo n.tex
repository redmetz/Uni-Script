\documentclass{scrartcl}
\usepackage{float}
\usepackage[utf8]{inputenc}
\usepackage[T1]{fontenc}
\usepackage{lmodern}
\usepackage[ngerman]{babel}
\usepackage{amsmath}
\usepackage{amssymb}
\usepackage{stmaryrd}
\usepackage{blindtext}
\usepackage{graphicx}
\title{26. Potenzieren modulo n}
\author{Adrian Hille}
\begin{document}
\Large 26. Potenzieren modulo n\\
\\
\normalsize
Wollen ausrechnen: \\
$2^{100000}~mod~100001$\\
Schreiben als Summe von Zweierpotenzen:\\
$100000 = 2^{2^{16}} + 2^{2^{15}}+ 2^{2^{10}} + 2^{2^{9}}+2^{2^{7}}+ 2^{2^5}$\\
Bin\"arschreibweise:\\
1100001101010000\\
Geschickt klammern:\\
$2^{10000} = 2^{2^{16}} \cdot 2^{2^{15}} \cdot 2^{2^{10}} \cdot 2^{2^{9}} \cdot 2^{2^{7}} \cdot 2^{2^5}$\\
TRICK: Zwischenergebnis modulo 100001 (Homomorphieregel)\\
\end{document}
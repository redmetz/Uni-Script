\documentclass{scrartcl}
\usepackage[utf8]{inputenc}
\usepackage[T1]{fontenc}
\usepackage{lmodern}
\usepackage[ngerman]{babel}
\usepackage{amsmath}
\usepackage{amssymb}
\usepackage{cancel}
\usepackage{graphicx}
\title{Lineare Algebra - Rechnen mit Matrizen}
\author{Tobias Raaf}
\begin{document}
	\Large Rechnen mit Matrizen \\\\
	\normalsize\\
	\begin{enumerate}
		\item Transponieren: Sei A=$(a_{ij})_{m\times n}\in K^{m \times n}$\\
		Dann ist $A^T$:=$(b_{ij})_{m\times n}\in K^{n\times m}$ mit $b_{ij}=a_{ji}$\\
		Beispiel: $A=\begin{pmatrix}
			a & b & c\\
			d & e & f\\
		\end{pmatrix}$, $A^T=\begin{pmatrix}
		a & d \\
		b & e \\
		c & f\\
		\end{pmatrix}$\\
		\item Skalarmultiplikation: Sei $k\in K$, k fest, $A \in K^{m\times n}$, $A=(a_{ij})_{m\times n}$.\\
		Dann ist $k*A$:=$(k*a_{ij})_{m\times n}\in K^{m\times n}$\\
		Beispiel: $\begin{pmatrix}
		1 & 2\\
		3 & 4\\
		\end{pmatrix}, k=\pi \Rightarrow k*A=\begin{pmatrix}
		\pi & 2\pi \\
		3\pi & 4\pi \\
		\end{pmatrix}$
		\item Addition: Sei A,B$\in K^{m\times n}$, A=$(a_{ij})_{m\times n}$, B=$(b_{ij})_{m\times n}$\\
		Dann ist A+B:=$(c_{ij})_{m\times n}\in K^{m\times n}$ mit $c_{ij}=a_{ij} + b_{ij}$\\
		Beispiel: A=$\begin{pmatrix}
	    1 & 2 & 3\\
	    4 & 5 & 6\\
	   	\end{pmatrix}$,B=$\begin{pmatrix}
	   	1 & 2 & 3\\
	   	4 & 5 & 6\\
	   	\end{pmatrix}\Rightarrow$ A+B=C=$\begin{pmatrix}
	   	2 & 4 & 6\\
	   	8 & 10 & 12\\
	   	\end{pmatrix}$\\
	   	Bemerkung: A-B:= $A+(-1)*B$, mit $(-1)\in K$\\
	   	\item Multiplikation: Sei $A\in K^{m\times n}$, $B\in K^{n\times p}$, A=$(a_{ij})_{m\times n}$, B=$(b_{ij})_{n\times p}$\\
	   	Dann ist $A*B:=(c_{ij})_{m\times p}$ mit $c_{ij}=\sum_{k=1}^{n}a_{ik}*b_{kj}$\\
	   	Beispiel: A=$\begin{pmatrix}
	   	1 & 2 & 3\\
	   	4 & 5 & 6\\
	   	\end{pmatrix}_{2\times 3}$, B=$\begin{pmatrix}
	   	a & d \\
	   	b & e \\
	   	c & f \\
	   	\end{pmatrix}_{3\times 2}$\\
	   	$(A*B)_{2\times 2}= \begin{pmatrix}
	   	1a+2b+3c & 1d+2e+3f\\
	   	4a+5b+6c & 4d+5e+6f\\
	   	\end{pmatrix}$
	   	
	\end{enumerate}
	\Large Matrixschreibweise für Lineare Gleichungssysteme\\\\
	\normalsize\\
	A=$(a_{ij})$ Koeffizientenmatrix\\
	(A|b) Erweiterte Koeffizientenmatrix\\
	$A*\begin{pmatrix}
	x_1\\
	...\\
	x_n\\
	\end{pmatrix}=b$\\
	A*x=b\\
\end{document}
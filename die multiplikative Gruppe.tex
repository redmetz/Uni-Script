\documentclass{scrartcl}
\usepackage{tabularx}
\usepackage{float}
\usepackage[utf8]{inputenc}
\usepackage[T1]{fontenc}
\usepackage{lmodern}
\usepackage[ngerman]{babel}
\usepackage{amsmath}
\usepackage{amssymb}
\usepackage{stmaryrd}
\usepackage{blindtext}
\usepackage{graphicx}
\title{32. die multiplikative Gruppe}
\author{Adrian Hille}
\begin{document}
\Large 32. Die multiplikative Gruppe\\
\\
\normalsize
Schreibweise:\\
$g \in G, n \in \mathbb{N}$\\
$g^n= g \circ ... \circ g$ (n-mal)\\
$g^{\circ}=e$ (neutrales Element)\\
$g^{n+1} := g^n \circ g$\\
$n \in \mathbb{Z}: g^n :=(g^{-n})^{-1} \quad (n <0)$\\
$\mathbb{Z}_6 \to 2$ hat kein Inverses\\
\\
Definition: Eine Zahl $a \in \mathbb{Z}_n \ \{0\}$ hei\ss t Nullteiler, wenn es ein $b \in \mathbb{Z}_n \ \{0\}$ gibt, sodass $a \cdot b = 0 (mod~n)$.\\
\\ 
Beispiel: 2 ist Nullteiler von $\mathbb{Z}_6$.\\
\\
Definition: Eine Zahl $a \in \mathbb{Z}_n$ hei\ss t Einheit, wenn es eine Zahl $b \in \mathbb{Z}_n$ gibt mit $a \cdot b \equiv 1 (mod~n)$.\\
\\
Lemma: Eine Zahl $m \in \mathbb{Z}_n \ \{0\}$ ist Einheit genau dann, wenn m und n teilerfremd.\\
Beweis: ($\Rightarrow$) Es gibt ein $b \in \mathbb{Z}_n$ mit $m \cdot b \equiv 1~mod~n$\\
$m \cdot b -1 = a \cdot n$ f\"ur alle $a \in \mathbb{Z}$\\
$1 = mb-an$\\
$\Rightarrow ggt(m,n)=1$\\
($\Leftarrow$) Seien m und n teilerfremd.\\
B\'{e}zout: $a \cdot m + b \cdot n = 1 \qquad (a,b \in \mathbb{Z} )$\\
Behauptung: a ist multiplikatives Inverses zu m\\
$m \cdot a = 1-bn$\\
$\equiv 1~mod~n \qquad \square$\\
Die Menge aller Einheiten bildet bzgl. der Multiplikation eine Gruppe $\mathbb{Z}_n^* (in~\mathbb{Z}_n)$\\
\begin{enumerate}
	\item Assoziativgesetz vererbt sich
	\item neutrales Element
	\item inverses Element: zu a gibt es b (Inverses) mit $a \cdot b \equiv 1~mod~n$
	\item $G^2 \mapsto G$
	\item a,b Einheiten zz.: $a \cdot b$ Einheit. Es gibt $a^{-1}, b^{-1}$ mit $aa^{-1} = 1 = bb^{-1}$ 
\end{enumerate}
Inverse: $(a \cdot b) \cdot (b^{-1} \cdot a^{-1}= \frac{a \cdot (bb^{-1}}a^{-1}{1}$\\
$= a a^{-1}$\\
$= 1$\\
\\
Inverses: $(a \cdot b)^{-1} := b^{-1}a^{-1}$\\
\\
\Large Eulersche $\Phi$-Funktion\\
\normalsize
Wie gro\ss $~$ ist die multiplikative Gruppe $\mathbb{Z}_n^*$ von $\mathbb{Z}$?\\
\begin{table}[H]
\begin{tabular}{l|l|l|l|l|l}
	n & 2 & 3 & 4 & 5 & 6 \\
	\hline
	$\Phi (n)$ & 1 & 2 & 2 & 4 & 2\\
\end{tabular}
\end{table}
$=\vert \mathbb{Z}_n^* \vert$
\\
Hat $n \in \mathbb{N}$ die Primfaktorzerlegung $n = p_1^{\alpha_1} \cdot (...) \cdot p_n^{\alpha_n} $\\
dann gilt: $\Phi (n) = (p_1-1)p_1^{\alpha_1-1} \cdot (...) \cdot (p_{n-1})p_n^{\alpha_n-1}$\\
\\
Beispiel:
$n = 6 = 2 (=p_1) \cdot 3(=p_2) $\\
$\Phi (6)=(2-1) \cdot 2^0 \cdot (3-1) \cdot 3^0 = 1 \cdot 2 \cdot 1 = 2$\\
\\
F\"ur kleine n einfache Formel. F\"ur gro\ss e n Problem, da Primfaktorzerlegung (noch) nicht berechenbar.\\


\end{document}
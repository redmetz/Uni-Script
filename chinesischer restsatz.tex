\documentclass{scrartcl}
\usepackage{float}
\usepackage[utf8]{inputenc}
\usepackage[T1]{fontenc}
\usepackage{lmodern}
\usepackage[ngerman]{babel}
\usepackage{amsmath}
\usepackage{amssymb}
\usepackage{stmaryrd}
\usepackage{blindtext}
\usepackage{graphicx}
\title{27. chinesischer Restsatzl}
\author{Adrian Hille}
\begin{document}
\Large 27. chinesischer Restsatz\\
\\
\normalsize
Satz: Es seien m, n teilerfremd und $k, l \in \mathbb{N}$. \\
Dann gilt es genau eine Zahl $x \in \{0, ..., mn-1 \}$ mit\\
$1 \equiv k~mod~m$\\
$1 \equiv l~mod~b$\\
\\
Beweis: Nach B\'ezout gibt es $a, b \in \mathbb{Z}$ dass $a \cdot m + b \cdot n = 1$.\\
Behauptung: $x := l \cdot a \cdot m + k \cdot b \cdot n$\\
leistet die gew\"unschte:\\
$lam+kbn \equiv (q-am)k \equiv k~mod~m$\\
$lam+kbn \equiv lam \equiv (1 - bn)l \equiv l~mod~n$\\
\end{document}
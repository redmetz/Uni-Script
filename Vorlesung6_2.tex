\documentclass{scrartcl}
\usepackage[utf8]{inputenc}
\usepackage[T1]{fontenc}
\usepackage{lmodern}
\usepackage[ngerman]{babel}
\usepackage{amsmath}
\usepackage{amssymb}
\usepackage{graphicx}
\title{Skript Vorlesung 4}
\author{Lukas Jährling}
\begin{document}
	\large
	VORLESUNG 6 NACH TITEL FRAGEN - Teil3
	\normalsize
	\\
	\\
	Definiton:\\
	Sei V ein K-VR, $T \subseteq V$ \\
	T heißt eine Basis von V, wenn gilt:\\
	\begin{enumerate}
		\item $Span(T)=V$ (T ist ein erzeugendes System von V)
		\item T ist linear unabhängig
	\end{enumerate}
	Beweis:\\
	eine Basis ($k_1, k_2, ..., k_n$) ist ein minimales erzeugendes System, denn $b_k \not \in Span(\{ k_1, ..., k_n  \}  \backslash \{ b_i  \})$\\ \\
	Anmerkung:\\
	$b_i \ in Span(...) \rightarrow$ es existiert\\
	$k_1,...,k_{i-1},k_{i+1},...,k_n$ mit $k_1\cdot b_1 + ... +k_{i-1} \cdot b_{i-1} + k_{i+1} \cdot b_{i+1} + k_n \cdot b_n = b_i$ \\
	$\rightarrow k_1 \cdot b_1 + ... + (-1)\cdot b + ... + k_n \cdot b_n = 0_v$ ????????????????  \\
	$\rightarrow (b_1,...,b_n)$ linear abhängig \\ \\
	Satz: \\
	Sei V ein K-VR und $B_1,B_2$ seine Basen dieses VR. Dann gilt: $\vert B_1 \vert = \vert B_2 \vert$ \\ \\
	Definition:\\
	Sei V ein K-VR und B eine Basis dieses VR.\\
	Gilt: $\vert B \vert = n \in \mathbb{N}$, dann ist V ein n-dimensionaler VR (dim V = n)\\
	Andernfalls ist V ein unendlich Dimensionaler VR (dim V = $\infty$) \\ \\
	Satz:\\
	Sei V ein K-VR und ($b_1,...,b_n$) eine Basis dieses VR. Dann existiert für jedes $v \in V$ eindeutig bestimmte $k_1,...,k_n \in K$ \\mit $v=k_1\cdot b_1 + ... + k_n\cdot b_n)$ (Basisdarstellung von V)\\
	Man nennt $V_{b_1,...,b_n}=\begin{pmatrix}
		k_1 \\ ... \\ k_n
	\end{pmatrix}_{b_1,...,b_n}$ den Koordinatenvektor von v bzgl. $(b_1,..,b_n)$
	
		
		
\end{document}
\documentclass{scrartcl}
\usepackage[utf8]{inputenc}
\usepackage[T1]{fontenc}
\usepackage{lmodern}
\usepackage[ngerman]{babel}
\usepackage{amsmath}
\usepackage{amssymb}
\usepackage{cancel}
\usepackage{graphicx}
\title{Lineare Algebra - Galois-Field(2)}
\author{Tobias Raaf}
\begin{document}
	\Large Matrizen über K\\\\
	\normalsize\\
	Eine Matrix A über K ist eine Abbildung.\\
	A:\{1,...,m\}x\{1,...,n\}$\Rightarrow$K:\{i,j\}$\mapsto a_{ij}$\\
	Dabei ist die Zeile der Matrix i und die Spalte j entsprechend. Das Element in Zeile i und Spalte j ist demnach $a_{ij}$.\\
	A$_{m\times n}=\begin{pmatrix}
	a_{11} & a_{12} & ... & a_{1n}\\
	a_{21} & a_{22} & ... & a_{2n}\\
	...    & ...    & ... & ...   \\
	a_{m1} & a_{m2} & ... & a_{mn}\\
	\end{pmatrix}_{m\times n}$ = $(a_{ij})_{m \times n}$\\
	\\
   \textbf{Spezielle Matrizen:}\\ 
   \begin{itemize}
   	\item Nullmatrix: $a_ij=0$ für alle ij, Bezeichnung: $0_{m\times n}$\\
   	\item Quadratische Matrix: m=n, die Diagonale mit den Elementen $a_{ij}$, mit i=j, heißt dabei Hauptdiagonale.\\
   	\item Diagonalmatrizen: Quadratische Matrizen mit $a_{ij}=0$, solange $i\neq j$.\\
   	\item Einheitsmatrix: Quadratische Matrix mit $a_{ij}=0$, solange $i\neq j$, und $a_{ij}=1$, wenn i=j.\\ 
   	
   \end{itemize}
	
\end{document}
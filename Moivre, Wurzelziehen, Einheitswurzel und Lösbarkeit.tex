\documentclass{scrartcl}
\usepackage[utf8]{inputenc}
\usepackage[T1]{fontenc}
\usepackage{lmodern}
\usepackage[ngerman]{babel}
\usepackage{amsmath}
\usepackage{amssymb}
\usepackage{cancel}
\usepackage{graphicx}
\title{Lineare Algebra - Satz von Moivre}
\author{Tobias Raaf}
\begin{document}
\Large \textbf{Satz von Moivre}\\\\
\normalsize
\textbf{Ein Sonderfall ist die Formel von Moivre:}\\
$z=e^{i\varphi}\Rightarrow z^n=1^n*e^{i*(n\varphi)}$\\
$\Leftrightarrow$ $z=e^{i\varphi}\Rightarrow z^n=e^{i*(n\varphi)}$\\\\
\textbf{Bemerkung:} Die Potenzen einer komplexen Zahl, welche auf dem Einheitskreis liegt, liegen auch auf dem Einheitskreis.\\\\
\textbf{Wurzelziehen in $\mathbb{C}$:}\\
1) Gesucht: Alle Lösungen von $z^n=1$\\
2) Gesucht: Alle Lösungen von $z^n=z_0$, für eine beliebige komplexe Zahl $z_0$\\
$\Rightarrow$ Ges.: $z,n\in\mathbb{N},z_0\in\mathbb{C}$\\
$z_0=r_0*e^{i\varphi_0}$\\\\
\textbf{Zu 1)}\\
Definition: Eine komplexe Zahl heißt n-te Einheitswurzel, wenn $z^n=1$ gilt.\\
Beispiel: i ist eine 4-te Einheitswurzel, denn: $i\in\mathbb{C},4\in\mathbb{N}$ und $i^4=i*i*i*i=(-1)*(-1)=1$\\
-i ist eine 4-te Einheitswurzel, denn: $-i\in\mathbb{C},4\in\mathbb{N}$ und $i^4=-i*-i*-i*-i=(-1)*(-1)=1$\\
i ist keine 2-te Einheitswurzel, denn: $i\in\mathbb{C},2\in\mathbb{N}$, aber $i^2=i*i=-1\neq1$\\
\\
\textbf{Ermitteln der n-ten Einheitswurzel}\\
$z_k=1*e^{i*\dfrac{2\pi}{n}*k},(k=\{0,1,...,n-1\})$ sind die n-ten Einheitswurzeln in $\mathbb{C}$\\\\
\textbf{Zu 2)}\\
$z^n=z_0$ $\Leftrightarrow$ $(r*e^{i\varphi})^n=r_0*e^{i\varphi_0}$\\
$\Leftrightarrow$ $r^n*e^{i*(n\varphi)}=r_0*e^{i\varphi_0}$\\
$\Leftrightarrow$ $r^n=r_0$ und $n\varphi=\varphi_0+k*2\pi$\\
$\Leftrightarrow$ $r=\sqrt[n]{r_0}$ und $\varphi=\dfrac{\varphi_0}{n}+k*\dfrac{2\pi}{n}$\\
$\Leftrightarrow$ $r=\sqrt[n]{r_0}$ und $\varphi\in\{\dfrac{\varphi_0}{n}+k*\dfrac{2\pi}{n}|k\in\{0,1,2,...,n-1\}\}$\\
Es gibt genau n Lösungen, nämlich:\\
$z_k=\sqrt[n]{r_0}*e^{i*(\dfrac{\varphi_0}{n}+k*\dfrac{2\pi}{n})}(k\in\{0,1,2,...,n-1\})$\\
\textbf{Bemerkung:} $z_k=\sqrt[n]{r_0}*e^{i*(\dfrac{\varphi_0}{n})}*e^{i*\dfrac{2\pi}{n}*k}
$\\
\hspace*{90pt}Spez. Lsg. von\hspace{15pt}n-te Einheits-\\
\hspace*{105pt} $z_n=z_0$\hspace{30pt}wurzel\\
denn: \\
$(\sqrt[n]{r_0}e^{i*\dfrac{\varphi_0}{n}})^n=(\sqrt[n]{r_0})^n*(e^{i*\dfrac{\varphi_0}{\cancel{n}}})^{\cancel{n}}=r_0*e^{i\varphi_0}=z_0$\\
\\
\textbf{Beispiel:}\\
Gesucht: $z\in\mathbb{C}$ mit $z^4=16$\\
Genau 4 Lösungen:\\
$z_0=2*1=2$\\
$z_1=2*i=2i$\\
$z_2=2*(-1)=-2$\\
$z_3=2*(-i)=-2i$\\
\\
\textbf{Lösbarkeit von Gleichungen in $\mathbb{C}$}\\
(1) $z^2=-1$ hat in $\mathbb{C}$ genau 2 Lösungen: i und -i\\
(2) $z^2=z_0\in\mathbb{C}$ hat in $\mathbb{C}$ genau 2 Lösungen\\
(3) $a*z^2+b*z+c=0$ hat in $\mathbb{C}$ genau 2 Lösungen\\
\textbf{Bemerkung:} $z^2+pz+q=0$\\
$z_{1,2}=\dfrac{-p}{2}\pm \sqrt{\dfrac{p^2}{2}-q}\Rightarrow$ Lösung von $x^2=\dfrac{-p}{4}-q$\\
$\sqrt{-1}\Rightarrow$ ist Lösung von $x^2=-1$\\
\end{document}
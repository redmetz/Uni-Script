\documentclass{scrartcl}\usepackage[utf8]{inputenc}
\usepackage[T1]{fontenc}
\usepackage{lmodern}
\usepackage[ngerman]{babel}
\usepackage{amsmath}
\usepackage{amssymb}
\usepackage{stmaryrd}
\usepackage{blindtext}
\title{�bersicht diskrete Strukturen}
\author{Adrian Hille}
\begin{document}
\large \textbf{Diskrete Strukturen}
\normalsize		\begin{enumerate}
			\item Die Symbole der Mengensprache
			\item Mengenangaben durch Aussondern
			\item Mengenoperationen
			\item Kodieren mit Mengen
			\item Doppeltes Abzählen
			\item Binomialkoeffizienten
			\item Die Russelsche Antinomie
			\item Die Axiome von Zermelo-Fraenkel
			\item Abbildungen
			\item Notation
			\item Der Satz von Cantor-Schröder-Bernstein
			\item Das Auswahlaxiom
			\item Die Kontinuumshypothese
			\item Permutationen
			\item Menge der natürlichen Zahlen
			\item Die Wohlordnung der natürlichen Zahlen
			\item Addition und Multiplikation 
			\item Teilbarkeit und Primzahlen
			\item Der euklidische Algorithmus
			\item Modulare Arithmetik
			\item Rechnen Modulo 2
			\item Rechnen Modulo 5
			\item Die Homomorphieregel
			\item Uhrzeiten
			\item Die letzten Ziffern
			\item Potenzieren modulo n
			\item Der chinesische Restsatz
			\item Zufall in der Informatik (1)
			\item Anwendung: Rechnen mit großen Zahlen
			\item Gruppen
			\item Beispiele für Gruppen
			\item Die multiplikative Gruppe
			\item Zyklische Gruppen
			\item Öffentlich ein Geheimnis vereinbaren
			\item Der Satz von Lagrange
			\item Das Lemma von Euler-Fermat
			\item Kryptographie mit öffentlichen Schlüsseln
			\item Graphen
			\item Knotenzusammenhang
			\item Färbbarkeit
			\item Bäume
			\item Zweifacher Zusammenhang
			\item Der Satz von Menger
			\item Eulersche Graphen
			\item Paarungen
		\end{enumerate}
\end{document}
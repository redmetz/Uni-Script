\documentclass{scrartcl}\usepackage[utf8]{inputenc}
\usepackage[T1]{fontenc}
\usepackage{lmodern}
\usepackage[ngerman]{babel}
\usepackage{amsmath}
\usepackage{amssymb}
\usepackage{stmaryrd}
\usepackage{blindtext}
\title{1- DIS- Die Symbole der Mengensprache}
\author{Adrian Hille}
\begin{document}
\Large 1. Die Symbole der Mengensprache
\\
\\
\normalsize
Mengen: \\ 
...jede Zusammenfassung M von bestimmten wohlunterschiedenen Objekten in unserer Anschauung oder des Denkens.\\
\\
Schreibweisen: 
   \begin{itemize}
    	\item e $\in$ M: e ist Element der Menge M.
    	\item e $\notin$ M: e ist kein Element der Menge M.
  	\item $\varnothing$: leere Menge.
    	\item A $\subseteq$ B: A ist Teilmenge von B.
    \end{itemize}

\Large 2. Mengenangaben durch Aussonderung
\\
\\
\normalsize
$\{x \mid Bedingung (x) \}$:  Die Mengen aller x, die die gegebene Bedingung erf\"ullen.\\
z.B.: alle geraden x
$\{x \mid \exists y \quad y\cdot 2=x) \}$
\\
  \begin{itemize}
    	\item $A \cap B$: Schnitt von A und B.
    	\item $A \cup B$: Vereinigung von A und B.
  	\item $A \setminus B$: Differenz von A und B.
    \end{itemize}

\Large 3. Mengenoperationen (Paar und Produkt)\\
\\
\normalsize
\textbf{geordnete Paare (a,b)}\\
(a,b) sind Menge der Gestalt $\{\{a\},\{a,b\}\}$\\
\\
$(a,b) \neq (b,a)$\\
$\{a,b\}=\{b,a\}$\\
$(a,b) = (c,d) \Rightarrow a=c und b=d$\\
$\{\{a\},\{a,b\}\}=\{\{c\},\{c,d\}\}$\\
\\
\Large 4. Kodieren mit Mengen\\
\\
\normalsize
\textbf{Produktmenge}\\
$A \times B$ zweier Mengen A,B wird definiert durch:\\
$A \times B\ := \{(a,b) \mid a \in A,b \in B\}$\\
Schreiben $\vert A\vert$ f\"ur die Anzahl der Elemente der Menge A. \\
\\
\textbf{Relationen}\\
Eine Teilmenge R von A und B hei\ss t (bin\"ar oder zweistellige) Relation.\\
\\
Falls A=B, so spricht man auch von einer zweistelligen Relation auf A.\\
\\
\textbf{Potenzmenge}\\
Die Potenzmenge von A, geschrieben $\mathcal{P}(A)$, ist die Menge aller Teilmengen von A.\\
Es gilt: $\vert \mathcal{P}(A) \vert = 2^{\vert A \vert}$\\
\\
\Large 5. Doppeltes Abz\"ahlen\\
\\
\normalsize
Das doppelte Abz\"ahlen ist ein kombinatorisches Beweisprinzip.\\
\\
Z\"ahle auf zweierlei Weise die Kardinalit\"at z von:\\
$\{(t_i , t_j) \mid t_i~gibt~ t_j~ die~ Hand\}$\\
$x_i$: Anzahl der Personen denen $t_i$ die Hand reicht\\
y: Anzahl aller Handschl\"age\\
Es gilt: \[ \sum_{i=1}^n = x_i=z=2y \]\\
und es folgt, dass eine gerade Anzahl der $x_i$ ungerade sein muss. $\square$ \\
\\
\Large 6. Binomialkoeffizienten \\
\\
\normalsize
$\binom{n}{k}$ (gesprochen: n \"uber k)\\
die Anzahl der k-elementigen Teilmengen einer n-elementigen Menge.\\
Bem.: \\
$\binom{n}{0} =1$\\
$\binom{n}{n}=1$\\ 
...sind Extremf\"alle\\
\\
Schreiben n! f\"ur $1 \cdot 2 \cdot ... \cdot n$, definieren $0! := 1$\\
\\
\textbf{Proposition}\\
F\"ur alle nat\"urlichen Zahlen n, k gilt: 
$\binom{n}{k} = \frac{n!}{k!(n-k)!}$\\
\\
Insbesondere gilt:\\
$\binom{n}{2} = \frac{n \cdot (n-1) }{2}$\\
\\
Wollen zeigen: $\binom{n}{k} = \frac{n!}{k!(n-k)!}$\\
Bilden k-elementige Teilmenge einer n-elementigen Menge: w\"ahle erstes Element, dann ein zweites usw., bis zum k-ten Element.\\
Daf\"ur gilt: $n \cdot (n-1) \cdot (...) \cdot (n-k+1) = \frac {n!}{(n-k!)}$ M\"oglichkeiten.\\
Auswahlreihenfolge spielt keine Rolle:\\
jeweils k! M\"oglichkeiten f\"uhren zur gleichen Teilmenge. $\square $\\
\\
Beobachtung: F\"ur alle nat\"urlichen Zahlen n,k gilt:\\
$\binom{n}{k}={n}{n \cdot k}$\\
(kombinatorischer Beweis:) Doppeltes Abz\"ahlen:\\
  \begin{enumerate}
    	\item Aus n Spielern einer Mannschaft mit k Spielern aufstellen.
    	\item Aus n Spielern n-k Spieler ausw\"ahlen, die nicht spielen.
    \end{enumerate}
(algebraischer Beweis:)\\
Folgt direkt aus $\binom{n}{k} = n \binom{n!}{k!(n-k)!} \qquad \square $\\
\\
Beobachtung: F\"ur alle nat\"urlichen Zahlen n,k gilt:\\
$k \binom{n}{k} = n \binom{n-1}{k-1}$\\
(kombinatorischer Beweis): Doppeltes Abz\"ahlen\\
  \begin{enumerate}
    	\item Aus n Spielern einer Mannschaft mit k Spielern inklusive Kapit\"an aufstellen.
    	\item Aus n Spielern einen Kapit\"an ausw\"ahlen und dann aus den \"ubrigen k-1 Spieler ausw\"ahlen
    \end{enumerate}
(algebraischer Beweis):\\
$k \binom{n}{k} = k \frac{n!}{k!(n-k)!} = \frac{n!}{(k-1)!(n-k)!} = n\frac{(n-1)!}{(k-1)!((n-1)-(k-1))!} = n \binom{n-1}{k-1} \qquad \square $ \\
\\
Beobachtung: F\"ur alle nat\"urlichen Zahlen n, k gilt: \\
\[ \sum_{k=0}^n =\binom{n}{k} = 2^n \]\\
Es gilt $2^n$ Teilmengen einer n-elementigen Menge.\\
\\
\textbf{Pascalsches Dreieck}\\
\begin{tabular}{rccccccccc}
$n=0$:&    &    &    &    &  1\\\noalign{\smallskip\smallskip}
$n=1$:&    &    &    &  1 &    &  1\\\noalign{\smallskip\smallskip}
$n=2$:&    &    &  1 &    &  2 &    &  1\\\noalign{\smallskip\smallskip}
$n=3$:&    &  1 &    &  3 &    &  3 &    &  1\\\noalign{\smallskip\smallskip}
$n=4$:&  1 &    &  4 &    &  6 &    &  4 &    &  1\\\noalign{\smallskip\smallskip}
\end{tabular}
\\
F\"ur alle nat\"urlichen Zahlen n, k gilt: \\
$\binom{n+1}{k}= \binom{n}{k-1}+\binom{n}{k}$\\
\\
Beweis: Sei M eine (n+1) elementige Menge und $x \in M $\\
w\"ahlen x aus $\binom{n}{k-1}$ M\"oglichkeiten.\\
w\"ahlen x nicht aus $\binom{n}{k}$ M\"oglichkeiten. \qquad $\square$\\
\\
\Large 7 Die Russelsche Antinomie\\
\\
\normalsize
allgemein: $\{x \mid Eigenschaften \}$\\
betrachte: $R := \{ x \mid x \notin x\}$\\
\\
Gilt $ R \in R$\\
   \begin{itemize}
    	\item Falls ja: Dann muss R die Aussonderungsbedingung erf\"uellen. (Eigenschaft gilt nicht f\"ur R $\to$ R $\notin$ R $\lightning$
	\item Falls nein: Eigenschaft erf\"ullt: R$\notin$ R $\to$ R $\in$ R $\lightning$\\
    \end{itemize}
\Large 8 Die Axiome von Zermelo-Fraenkel \\
\\
\normalsize
       \begin{itemize}
    	\item Extensionalit\"at: z.B.: $\{a,b\}=\{b, a, b\}$
	\item Vereinigung: z.B.: $\{ \{a, b\} \{c , d, a\} \}$
	\item unendliche Mengen: z.B.: $\{\{ \varnothing\}\{\{ \varnothing\}\}\{\{\{ \varnothing\}\}\}\}$
	\item Fundierung: ($x \notin x$: zum selber knobeln)
    \end{itemize}



\end{document}  
\documentclass{scrartcl}
\usepackage[utf8]{inputenc}
\usepackage[T1]{fontenc}
\usepackage{lmodern}
\usepackage[ngerman]{babel}
\usepackage{amsmath}
\usepackage{amssymb}
\title{RSA}
\author{Tobias Raaf}
\begin{document}
	\large \textbf{RSA}\\\\
	\normalsize\\ \begin{enumerate}
		\item Schlüssel anlegen\\
		Bob wählt zwei große Primzahlen p und q.\\
		Berechnet n:=p$\cdot$q\\
		Wähle eine von $\phi(n)$ teilerfremde Zahl d.($\phi(n)=(p-1)\cdot(q-1))$\\
		$\Rightarrow$ Es gibt ein Inverses i von d in $|\mathbb{Z}_{n}^{*}|$\\
		$d\cdot i\equiv 1 (mod~\phi(n))$\\
		Anmerkung: i kann mit dem erweiterten euklidischen Algorithmus  errechnet werden.\\
		n,i werden öffentlich bekannt gegeben.\\
		\item Alice: $m\in \\mathbb{Z}_n$ Nachricht\\
		c:=$m^{i}~mod~n$ (Al Kaschi)\\
		c wird an Bob geschickt.\\
		\item Entschlüsseln: Bob berechnet $c^{d}~mod~n$\\
		Behauptung: $m\equiv c^d$ (mod n)\\
		$c^d=(m^{i})^{d}=m^{i\cdot d}\equiv m^{1+n\cdot \phi(n)}$ mod n\\
		$id\equiv 1~mod ~ \phi(n)$,  $m^{\phi(n)}\equiv 1$ (mod n)\\
	\end{enumerate}
\end{document}
\documentclass{scrartcl}
\usepackage[utf8]{inputenc}
\usepackage[T1]{fontenc}
\usepackage{lmodern}
\usepackage[ngerman]{babel}
\usepackage{amsmath}
\usepackage{amssymb}
\title{Skript Vorlesung 4}
\author{Lukas Jährling}
\begin{document}
	\large
	Beispiel aus der Codierungstheorie
	\\
	\\
	\\
	\normalsize
	$
	H = \begin{pmatrix}
		1 & 1 & 1 & 1 & 0 & 0 & 0 \\
		1 & 1 & 0 & 0 & 1 & 1 & 0 \\
		1 & 0 & 1 & 0 & 1 & 0 & 1
	\end{pmatrix}
	$
	ist die Kontrollmatrix für $(7,4)$ Haming-Code
	\\
	\\
	$
	\begin{pmatrix}
		1 \\ 1 \\ 1 \\ 1 \\ 1 \\ 1 \\ 1
	\end{pmatrix}
	$
	ist ein Codewort, denn H $\cdot \begin{pmatrix}
		1 \\ 1 \\ 1 \\ 1 \\ 1 \\ 1 \\ 1
	\end{pmatrix}
	=
	\begin{pmatrix}
		1 \cdot 1 + 1 \cdot 1 + 1 \cdot 1 + 1 \cdot 1 + 0 \cdot 1 + 0 \cdot 1 + 0 \cdot 1
		\\ ...
		\\...
		\\
	\end{pmatrix}
	=
	\begin{pmatrix}
		0 \\ 0 \\ 0
	\end{pmatrix}
	$
	\\
	\\
	da $
	\begin{tabular}{c|c c}
		+ & 0 & 1
		\\ \hline
		0 & 0 & 1
		\\
		1 & 1 & 0
	\end{tabular}
	$
	und
	$
	\begin{tabular}{c|c c}
		$\cdot$ & 0 & 1
		\\ \hline
		0 & 0 & 0
		\\
		1 & 0 & 1
	\end{tabular}
	$
	\\
	\\
	\\
	\\
	noch ein Beispiel
	\\
	\\
	$
	\begin{pmatrix}
		1 \\ 1 \\ 1 \\ 0 \\ 1 \\ 1 \\ 1
	\end{pmatrix}
	$
	ist kein Codewort, denn 
	$H \cdot
	\begin{pmatrix}
		1 \\ 1 \\ 1 \\ 0 \\ 1 \\ 1 \\ 1
	\end{pmatrix}
	=
	\begin{pmatrix}
		1 \cdot 1 + 1 \cdot 1 + 1 \cdot 1 + 0 \cdot 1 + 1 \cdot 1 + 1 \cdot 1 + 1 \cdot 1
		\\
		...
		\\
		...
	\end{pmatrix}
	=
	\begin{pmatrix}
		1 \\ 0 \\ 0
	\end{pmatrix}
	$
\end{document}


\documentclass{scrartcl}
\usepackage[utf8]{inputenc}
\usepackage[T1]{fontenc}
\usepackage{lmodern}
\usepackage[ngerman]{babel}
\usepackage{amsmath}
\usepackage{amssymb}
\title{Skript Kapitel}
\author{Tobias Raaf}
\begin{document}
	\large \textbf{Lineare Algebra}
\normalsize	\begin{enumerate}
		\item Definition Körper der komplexen Zahlen
		\item Rechenregeln in den komplexen Zahlen
		\item Gaußsche Zahlenebene
		\item arithmetische, trigonometrische und eulersche Darstellung
		\item Multiplikation und Division in der eulerschen Darstellung
		\item Komplexe Uhr und Satz von Moivre
		\item Lösbarkeit von Gleichungen in $\mathbb{C}$
		\item Galois Field(2)
		\item Lineare Gleichungssysteme über Körpern
		\item Die Lösungsmengen von LGS
		\item Matrizen über K und spezielle Matrizen
		\item Rechnen mit Matrizen
		\item (erweitere) Koeffizientenmatrix
		\item Matrizenring
		\item Hamming-Code
		\item Elementare Zeilenumformungen und Stufenformen
		\item Gauss-Algorithmus
		\end{enumerate}
		\large \textbf{Diskrete Strukturen}
\normalsize		\begin{enumerate}
			\item Die Symbole der Mengensprache
			\item Mengenangaben durch Aussondern
			\item Mengenoperationen
			\item Kodieren mit Mengen
			\item Doppeltes Abzählen
			\item Binomialkoeffizienten
			\item Die Russelsche Antinomie
			\item Die Axiome von Zermelo-Fraenkel
			\item Abbildungen
			\item Notation
			\item Der Satz von Cantor-Schröder-Bernstein
			\item Das Auswahlaxiom
			\item Die Kontinuumshypothese
			\item Permutationen
			\item Menge der natürlichen Zahlen
			\item Die Wohlordnung der natürlichen Zahlen
			\item Addition und Multiplikation 
			\item Teilbarkeit und Primzahlen
			\item Der euklidische Algorithmus
			\item Modulare Arithmetik
			\item Rechnen Modulo 2
			\item Rechnen Modulo 5
			\item Die Homomorphieregel
			\item Uhrzeiten
			\item Die letzten Ziffern
			\item Potenzieren modulo n
			\item Der chinesische Restsatz
			\item Zufall in der Informatik (1)
			\item Anwendung: Rechnen mit großen Zahlen
			\item Gruppen
			\item Beispiele für Gruppen
			\item Die multiplikative Gruppe
			\item Zyklische Gruppen
			\item Öffentlich ein Geheimnis vereinbaren
			\item Der Satz von Lagrange
			\item Das Lemma von Euler-Fermat
			\item Kryptographie mit öffentlichen Schlüsseln
			\item Graphen
			\item Knotenzusammenhang
			\item Färbbarkeit
			\item Bäume
			\item Zweifacher Zusammenhang
			\item Der Satz von Menger
			\item Eulersche Graphen
			\item Paarungen
		\end{enumerate}
\end{document}
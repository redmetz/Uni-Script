\documentclass{scrartcl}
\usepackage{tabularx}
\usepackage{float}
\usepackage[utf8]{inputenc}
\usepackage[T1]{fontenc}
\usepackage{lmodern}
\usepackage[ngerman]{babel}
\usepackage{amsmath}
\usepackage{amssymb}
\usepackage{stmaryrd}
\usepackage{blindtext}
\usepackage{graphicx}
\title{25. die letzten Ziffern}
\author{Adrian Hille}
\begin{document}
\Large 25. die letzen Ziffern\\
\\
\normalsize
$333333 \cdot 444444 \cdot 56789 = N$\\
Frage: Was sind die letzten beiden Ziffern von N?\\
L\"osung: Rechnen modulo 100!\\
\\
$N \equiv 33 \cdot 44 \cdot 89$ (mod n)\\
$N \equiv 33 \cdot 11 \cdot 4 \cdot 89$ (mod n)\\
$N \equiv (330+33)(320 \cdot 36)$ (mod n)\\
$N \equiv 63 \cdot 56$ (mod n)\\
$N \equiv 28$ (mod n)\\
\\
\end{document}
\documentclass{scrartcl}
\usepackage[utf8]{inputenc}
\usepackage[T1]{fontenc}
\usepackage{lmodern}
\usepackage[ngerman]{babel}
\usepackage{amsmath}
\usepackage{amssymb}
\title{Skript Vorlesung 4}
\author{Lukas Jährling}
\begin{document}
	\large
	Matrizenring
	\\
	\\
	\\
	\normalsize
	Beweis ($K^{nxn};+;\cdot$) bildet einen Ring (Matrizenring)
	\\
	Aber keinen Körper $\rightarrow$ da man nicht durch beliebige Matrizen dividieren kann
	\\
	\\
	Beweis: Man kann nicht durch die Nullmatrix dividieren, denn \\
	$A \cdot 0_{n\times m} = B_{n\times n} \cdot 0_{n\times n} \not\Rightarrow A = B \hspace{1cm} (A \in K^{n\times n} ) $
	\\
	\\
	Beweis: Es gibt auch Matrizen, die nicht die Nullmatrix sind, und durch die man nicht dividieren kann. z.B.:
	\\
	
	\hspace{0cm}
	$\begin{pmatrix}
		0 & 1 \\ 0 & 1
	\end{pmatrix}$ denn:
	$\begin{pmatrix} 
		0 & 0 \\ 0 & 0
	\end{pmatrix}
	=
	\begin{pmatrix}
		1 & 0 \\ 0 & 0
	\end{pmatrix}
	\cdot
	\begin{pmatrix}
		0 & 0 \\ 1  & 1
	\end{pmatrix}
	=
	\begin{pmatrix}
		0 & 0 \\ 1 & 0
	\end{pmatrix}
	\cdot
	\begin{pmatrix}
		0 & 0 \\ 1 & 1
	\end{pmatrix}
	$
	und
	$
	\begin{pmatrix}
		1 & 0 \\ 0 & 0
	\end{pmatrix}
	\not = 
	\begin{pmatrix}
		0 & 0 \\ 1 & 0
	\end{pmatrix}$
	\\
	\\
	\\
	Um durch $A \in K^{n\times n} $dividieren zu können, muss die inverse Matrix $A^{-1}$ existieren.
\end{document}


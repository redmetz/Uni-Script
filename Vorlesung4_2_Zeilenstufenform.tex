\documentclass{scrartcl}
\usepackage[utf8]{inputenc}
\usepackage[T1]{fontenc}
\usepackage{lmodern}
\usepackage[ngerman]{babel}
\usepackage{amsmath}
\usepackage{amssymb}
\title{Skript Vorlesung 4}
\author{Lukas Jährling}
\begin{document}
	\large
	reduzierte Zeilenstufenform eines LGS
	\\
	\\
	\\
	\normalsize
	Lösungsmenge eines LGS notieren, wenn es in reduzierter Zeilenstufenform gegeben ist.
	\\
	\\
	$
	\begin{tabular}{c c c c c c c|c}
		1 & 0 & -7 & 2 & 0 & 1 & 0 & 0
		\\
		0 & 1 & 2 & -3 & 0 & -5 & 0 & 8
		\\
		0 & 0 & 0 & 0 & 1 & 5 & 0 & 15
		\\
		0 & 0 & 0 & 0 & 0 & 0 & 1 & 22
		\\
		0 & 0 & 0 & 0 & 0 & 0 & 0 & 1
	\end{tabular}
	\rightarrow
	\begin{pmatrix}
		x_1 \\ x_2 \\ x_3 \\ x_4 \\ x_5 \\ x_6 \\ x_7
	\end{pmatrix}
	=
	\begin{pmatrix}
		0 -7\cdot r - 2\cdot s - t
		\\
		8 -2\cdot r + 3\cdot s + 5\cdot t
		\\
		r
		\\
		s
		\\
		15 - 5\cdot t
		\\
		t
		\\
		42
	\end{pmatrix}
	$
	$
	r, s, t \in \mathbb{R}
	$
	\\
	\\
	Allgemein gilt:
	\\
	\\
	---Matrix---
	\\
	\\
	$
	a_i \not = 0 $ für $ i=1,...r
	$
	\\
	\\In dieser Form ist eine Entscheidung möglich ob die Lösungsmenge eine leere Menge ist. $L = {\o}$ genau dann wenn ein $i \in \{r+1,...,m\}$ existiert so, dass $b_i \not = 0$
	\\
	\\
	Elementare Zeilenumformungen
	\begin{enumerate}
		\item Vertauschen zweier Zeilen
		\item Multiplizieren einer Gleichung mit einem Faktor $k\in K\{0\}$
		\item Addieren des k-fachen einer anderen Zeile in eine Zeile $(k\in K)$
	
	\end{enumerate}
	
	Satz: Elementare Zeilenumformungen ändern die Lösungsmenge des LGS nicht.
	\\
	\\
	\\
	Beweis für 2)
	\\
	\\
	BILD BILD BILD BILD BILD\\
	2 Stück zur Umformung
	\\
	\\
	Algorithmus
	\\
	\\
	BILD BILD BILD BILD\\
	2 zur Zeilenstufenform und zur reduzierten Zeilenstufenform
	\\
	\\
	Beweis: Bei LGS in ZSF kann entschieden werden ob die Lösungsmenge leer ist
	\\
	Be: Beim erzeugen der ZSF mittels elementarer Zeilenstufenumformungen darf man Spalten vertauschen, wenn man die Variablennamen ...
	\\
	\\
	BILD BILD BILD BILD BILD
	mit Bsp
	\\
\end{document}


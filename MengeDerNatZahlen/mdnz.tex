\documentclass{scrartcl}
\newtheorem{defi}{Definition}
\newtheorem{satz}{Satz}
\newtheorem{bsp}{Beispiel}
\newtheorem{lem}{Lemma}
\usepackage[utf8]{inputenc}
\usepackage[T1]{fontenc}
\usepackage{lmodern}
\usepackage[german]{babel}
\usepackage{amsmath}
\usepackage{amssymb}
\usepackage{graphicx}


\begin{document}

\section{Die Menge der nat\"urlichen Zahlen}
F\"ur eine Menge M definiere $M^+=M\cup\{M\}.$\\

\subsection{Die Wohlordnung der natürlichen Zahlen}
Für $n,m \in \mathbb{N}$ gilt $n < m$ genau dann, wenn $n \in m$. Wir schreiben $n \leq m$ falls $n < m$ oder $n = m$ gilt. Die Relation $\leq$ ist eine \emph{Wohlordnung:} Für jede Teilmenge $T$ von $\mathbb{N}$ existiert ein \emph{kleinstes Element}. Das heißt für jedes $T \subseteq \mathbb{N}$ gibt es ein Element $x \in T$, so dass es kein $y\in T$ gibt mit $y < x$. Wir bemerken, dass < und $\leq$ binäre Relation auf $\mathbb{N}$ sind, weshalb wir $\leq$ als die Teilmenge von $\mathbb{N} \times \mathbb{N}$ betrachten, die alle geordneten Paare $(m,n)$ enthält mit $n \leq m$. 
\\
\subsection{Addition und Multiplikation}
Die Addition ist eine Funktion $+:=\mathbb{N} \times \mathbb{N} \to \mathbb{N} $(= eine ``zweistellige'' Funktion auf $\mathbb{N}$) und wird wiefolgt induktiv definiert:
\begin{center}
	$n+0:=n$
	\\
	$n+m^+:=(n+m)^+$
\end{center}
Auch die Multiplikation $\cdot:\mathbb{N} \times \mathbb{N} \to \mathbb{N}$ kann mein induktiv definieren:
\begin{center}
	$n\cdot 0 := 0$
	\\
	$n\cdot m^+ :=n\cdot m+n$
\end{center}
Und zum Schluss betrachten wir noch die Exponentation $\mathbb{N} \times \mathbb{N} \to \mathbb{N}$ mit Hilfe der Multiplikation:
\begin{center}
	$n^0:=1$
	\\
	$n^{m^{+}}:=n^mn$
\end{center}
\subsection{Teilbarkeit und Primzahlen}
Wir definieren auf $\mathbb{N}$ die \emph{Teilbarkeitsrelation:} für $a,b \in \mathbb{N}$ gelte $a\mid b$(sprich a \emph{teilt} b) genau dann, wenn es ein $k \in \mathbb{N}$ gibt mit $a \cdot k = b$. In diesem Fall heißt \emph{a Teiler} von $b$. 
\begin{defi}Eine Zahl $p\in \mathbb{N}$ heißt Primzahl(oder prim), wenn sie größer als 1 ist und nur durch 1 und sich selbst teilbar ist. Ein Primteiler von n ist ein Teiler von n, der prim ist. 
\end{defi}
\begin{satz}(Fundamentalsatz der Arithmetik). Jede natürliche Zahl n > 0 kann auf genau eine Weise als Produkt 
\begin{center}$n=p_{1}^{\alpha_{1}} \cdot p_{2}^{\alpha_{2}} \cdot ... \cdot p_{k}^{\alpha_{k}}$\end{center}
geschrieben werden, wobei $k \in \mathbb{N}, p_1 < p_2 < ... < p_k$ Primzahlen, und 
$\alpha_1, \alpha_2, ..., \alpha_k \in \mathbb{N}$ größer als 1 sind. 
\end{satz}
\subsection{Der euklidische Algorithmus}
Der euklidische Algorithmus ist ein effizientes Verfahren, um den größten gemeinsamen Teiler zweier Zahlen zu berechen. \\
Der \emph{größte gemeinsame Teiler} von $a,b \in \mathbb{N}$ ist die größte natürliche Zahl $d$, die $a$ und $b$ teilt. Wir schreiben $ggT(a,b)$ für diese Zahl $d$. 
\begin{lem}(Division mit Rest). Seien $a,b \in \mathbb{Z}$ und $b!=0$. Dann gibt es $q,r \in \mathbb{Z}$ mit $a=qb+r$ und $0\leq r < |b|$.
\end{lem}
Für die Zahl r aus dem Lemma schreiben wir auch $a mod b$; was wir schon unter dem \emph{Rest} aus der schriftlichen Division her kennen. Für $q \in \mathbb{Q}$ schreiben wir $\left\rfloor{q}\right\rfloor$ für die eindeutige größte Zahl $z \in \mathbb{Z}$ die kleiner ist als $q$. Dann gilt für $a,b \in \mathbb{N}$ und $b!=0$ dass $a=\left\rfloor{a\setminus b}\right\rfloor + a mod b.$

\begin{lem}
Es seien $a,b \in \mathbb{N}$ mit $b > 0$. Dann gilt $ggT(a,b)=ggT(b, a mod b)$.
\end{lem}
Dieses Lemma ist die zentrale Beobachtung für die Korrektheit für den euklidischen Algorithmus:
\\
//Eingabe: $m,n \in \mathbb{N}$ mit $m\leq n$\\
//Ausgabe: $ggT(m,n)$. \\
Falls $m\mid n$\\
\hspace{4pt}gebe $m$ aus\\
ansonsten\\
\hspace{4pt}gebe $EUKLID(n\mod{m},m)$ aus.
\subsection{Erweiterter euklidischer Algorithmus}
Durch eine kleiner Erweiterung kann der euklidische Algorithmus auch dazu verwendet werden, um für gegeben $m,n \in \mathbb{N}$ die Zahlen $a,b \in \mathbb{Z}$ aus dem Lemma von Bézout zu berechnen. 
\begin{lem}
Es seien $m,n \in \mathbb{N}$ nicht beide 0. Dann gibt es ganze Zahlen $a,b \in \mathbb{Z}$ mit $ggT(m,n)=am+bn$.
\end{lem}
Erweiterter Algorithmus:\\
//Der erweiterte euklidische Algorithmus E-EUKLID$(m,n)$\\
//Eingabe: $m,n \in \mathbb{N}$ mit $m\leq n$.\\
//Ausgabe: $a,b \in \mathbb{Z}$ so dass $ggT(m,n)=am+bn$\\
Falls $m\mid n$\\
\hspace{4pt}gebe $(1,0)$ aus.\\
ansonsten\\
\hspace{4pt}Sei $(b',a')$ die Ausgabe von E-EUKLID$(n\mod{m},m)$.\\
\hspace{4pt}Gebe $(a'-b'\left\rfloor{n\setminus m}\right\rfloor, b')$ aus.\\

\begin{lem}(Lemma von Euklid). Teilt eine Primzahl das Produkt zweier natürlicher Zahlen, so auch mindestens einen der Faktoren.
\end{lem}

\end{document}

\documentclass{scrartcl}
\usepackage{float}
\usepackage[utf8]{inputenc}
\usepackage[T1]{fontenc}
\usepackage{lmodern}
\usepackage[ngerman]{babel}
\usepackage{amsmath}
\usepackage{amssymb}
\usepackage{stmaryrd}
\usepackage{blindtext}
\usepackage{graphicx}
\title{24. Uhrzeiten}
\author{Adrian Hille}
\begin{document}
\Large 24. Uhrzeiten\\
\\
\normalsize
Eine gr\"o \"ss ere Menge an Bauschotter wird mit einer Eisenbahn von A nach B transportiert, daf\"ur sind 50 Fahrten erforderlich. Das Beladen des Zuges dauert 4 Stunden, jede Fahrt 2 Stunden pro Richtung und das Abladen 3 Stunden. Pausen werden nicht gemacht. Mit dem Beladen f\"ur die erste Fahrt wurde mittags um 12 Uhr begonnen. Zu welcher Uhrzeit wird der letzte Zug wieder erwartet?\\
\\
Antwort: F\"ur jede Fahrt wird vom Beginn des Beladens bis zur R\"uckkehr ein Zeitraum von 11 Stunden ben\"otigt, insgesamt also 50 x 11 Stunden. Da nur nach der Uhreit der R\"uckkehr gefragt ist, kann modulo 24 gerechnet werden.\\
$12+50 \cdot 11 \equiv 12+2 \cdot 11 \equiv 34 \equiv 10~(mod~24)$\\
Man erh\"alt, dass der Zug zehn Stunden nach Mitternacht ankommt.
\end{document}
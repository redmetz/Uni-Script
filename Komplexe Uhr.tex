\documentclass{scrartcl}
\usepackage[utf8]{inputenc}
\usepackage[T1]{fontenc}
\usepackage{lmodern}
\usepackage[ngerman]{babel}
\usepackage{amsmath}
\usepackage{amssymb}
\usepackage{cancel}
\usepackage{graphicx}
\title{Lineare Algebra - Komplexe Uhr}
\author{Tobias Raaf}
\begin{document}
\Large \textbf{Komplexe Uhr}\\\\
\normalsize
\includegraphics{C:/Users/Tobias/Desktop/Git-Projekte/Skript/Mathe/KomplexeUhr}
\\\\\\
\textbf{Fragestellung:} Für welchen Winkel ist die Uhrzeit Minuten-Stunden-Symmetrisch? (Bsp. 12:12 Uhr)\\
$z_g$ bilde den großen Zeiger ab, $z_k$ bilde den kleinen Zeiger ab.\\
Also: Für welchen Winkel $\varphi$ ist $z_g=z_k$?\\
$z_k=1*e^{i12\varphi}$\\
$z_g=1*e^{i\varphi}$\\
$\Rightarrow$$z_k=z_g$ $\Leftrightarrow$ $1*e^{i\varphi}=1*e^{i12\varphi}$\\
$\Leftrightarrow$ $12\varphi=\varphi$ oder $12\varphi=\varphi+2\pi$, oder...\\
$\Leftrightarrow$ $12\varphi=\varphi +k*2\pi$\\
$\Leftrightarrow$ $11\varphi=k*2\pi$\\
$\Leftrightarrow$ $\varphi=k*\dfrac{2\pi}{11}$\\
$\Leftrightarrow$ $\varphi\in\{0*\dfrac{2\pi}{11}, 1*\dfrac{2\pi}{11}, 2*\dfrac{2\pi}{11}, 3*\dfrac{2\pi}{11},...,10*\dfrac{2\pi}{11},\cancel{11*\dfrac{2\pi}{11}}\}$\\
$\Rightarrow$ $z^1=r*e^{i\varphi}$\\
$z^2=r^2*e^{2i\varphi}$\\
$z^3=r^3*e^{3i\varphi}$\\
\textbf{Allgemein gilt:}$z^n=r^n*e^{ni\varphi}$, gilt für alle $n\in\mathbb{N}$\\
\end{document}
\documentclass{scrartcl}
\usepackage[utf8]{inputenc}
\usepackage[T1]{fontenc}
\usepackage{lmodern}
\usepackage[ngerman]{babel}
\usepackage{amsmath}
\usepackage{amssymb}
\usepackage{cancel}
\usepackage{graphicx}
\title{Lineare Algebra - Galois-Field(2)}
\author{Tobias Raaf}
\begin{document}
	\Large Lineare Gleichungssysteme über Körpern (LGS)\\\\
	\normalsize\\
	Es sei K ein Körper.\\
	Form einer Linearen Gleichung mit n Unbekannten über K: $a_1*x_1*a_2*x_2*...*a_n*x_n=b$\\
	Die Unbekannten sind dann: $x_1,x_2,...,x_n$\\
	Die Koeffizienten: $a_1,a_2,...,a_n\in K$\\
	Das Absolutglied: b $\in$ K\\
	\textbf{Wichtig: Ein LGS heißt linear, da es keine Unbekanntenmultiplikation gibt!}\\
	\\
	Kurzform: $\sum_{j=1}^{n}a_{j}*x_{j}= b$\\
	Konkreter ein LGS über K mit m Gleichungen und n Unbekannten:\\
	$\sum_{j=1}^{n}a_{ij}*x_j=b_j$ (i=1,2,...,m)\\
	Lösung: $(l_1,l_2,...,l_n)$ mit $l_1,l_2,...,l_n\in K$ heißt Lösung.\\
	Dabei ist $(l_1,l_2,...,l_n)$ das Lösungstupel, mit der Voraussetzung, dass $\sum_{j=1}^{n}a_{ij}*l_j=b_i$ gilt.\\
	Die Lösungsmenge $\mathbb{L}$ ist die Menge aller Lösungen.\\
	\Large Arten von LGS\\\\
	\normalsize\\
	\textbf{(1) Das homogene LGS:} (0,0,0,...,0) ist \textbf{eine} geltende Lösung.\\
	\textbf{(2)  inhomogene LGS:} (0,0,0,...,0) ist \textbf{keine} Lösung, d.h. es existiert ein $i\in{1,2,...,m}$ mit $b_\neq0$\\
	
	\end{document}
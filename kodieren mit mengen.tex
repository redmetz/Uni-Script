\documentclass{scrartcl}\usepackage[utf8]{inputenc}
\usepackage[T1]{fontenc}
\usepackage{lmodern}
\usepackage[ngerman]{babel}
\usepackage{amsmath}
\usepackage{amssymb}
\usepackage{stmaryrd}
\usepackage{blindtext}
\title{1- DIS- Die Symbole der Mengensprache}
\author{Adrian Hille}
\begin{document}
\Large 4. Kodieren mit Mengen\\
\\
\normalsize
\textbf{Produktmenge}\\
$A \times B$ zweier Mengen A,B wird definiert durch:\\
$A \times B\ := \{(a,b) \mid a \in A,b \in B\}$\\
Schreiben $\vert A\vert$ f\"ur die Anzahl der Elemente der Menge A. \\
\\
\textbf{Relationen}\\
Eine Teilmenge R von A und B hei\ss t (bin\"ar oder zweistellige) Relation.\\
\\
Falls A=B, so spricht man auch von einer zweistelligen Relation auf A.\\
\\
\textbf{Potenzmenge}\\
Die Potenzmenge von A, geschrieben $\mathcal{P}(A)$, ist die Menge aller Teilmengen von A.\\
Es gilt: $\vert \mathcal{P}(A) \vert = 2^{\vert A \vert}$\\
\end{document}

\documentclass{scrartcl}
\usepackage[utf8]{inputenc}
\usepackage[T1]{fontenc}
\usepackage{lmodern}
\usepackage[ngerman]{babel}
\usepackage{amsmath}
\usepackage{amssymb}
\title{Der Kleine Fermat und Satz von Euler-Fermat}
\author{Tobias Raaf}
\begin{document}
	\large \textbf{Der kleine Fermat}\\\\
	\normalsize\\ Sei p prim und sei $a\in \mathbb{Z}$, sodass p kein Teiler von a ist, dann:\\
	$a^{p-1}\equiv 1 (mod~p)$\\\\
	\textbf{Beweis:} $|\mathbb{Z}_{p}^{*}|=p-1$\\
	b:=(a mod p)$\in |\mathbb{Z}_{p}^{*}|$\\
	$b^{|\mathbb{Z}_{p}^{*}|}$=1\\
	$a^{p-1}\equiv 1(mod~p)$\\
	\\\large \textbf{Der Satz von Euler-Fermat}\\\\
	\normalsize\\Der kleine Fermat nur ohne die Bedingung, dass $|\mathbb{Z}_{p}^{*}|$, mit p prim.\\ 
	\textbf{Satz:} Sei nun b beliebig und ggT(a,n)=1\\
	Dann: $a^{\phi(n)}\equiv 1 (mod~n)$\\
	$|\mathbb{Z}_{n}^{*}|=\phi(n)$\\
	b:= (a mod n)$\in$$|\mathbb{Z}_{n}^{*}|$\\
	$b^{\phi(n)}= 1$\\
	$a^{\phi(n)}\equiv 1(mod~n)$\\
	Der  Satz findet Anwendung innerhalb des Verschlüsselungsverfahren RSA.
\end{document}
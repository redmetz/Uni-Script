\documentclass{scrartcl}
\usepackage[utf8]{inputenc}
\usepackage[T1]{fontenc}
\usepackage{lmodern}
\usepackage[ngerman]{babel}
\usepackage{amsmath}
\usepackage{amssymb}
\usepackage{graphicx}
\title{Skript Vorlesung 4}
\author{Lukas Jährling}
\begin{document}
	\large
	VORLESUNG 6 NACH TITEL FRAGEN
	\normalsize
	\\
	\\
	Bsp. $K=GF(2)$, $V=GF(2)$ ist ein GF(2)-VR\\
	Jeder Untervektorraum (UVR) von $GF(2)^n$ wird Linearcode genannt\\\\
	z.B.: $U=\{\begin{pmatrix}
		v_1\\...\\v_7
		\end{pmatrix} \in GF(2)^7 \vert \begin{pmatrix}
		1 & 1 & 1 & 1 & 0 & 0 & 0 \\
		1 & 1 & 0 & 0 & 1 & 1 & 0 \\
		1 & 0 & 1 & 0 & 1 & 0 & 1
		\end{pmatrix} \cdot \begin{pmatrix}
		v_1 \\ ... \\ v_7
		\end{pmatrix} = \begin{pmatrix}
		0 \\ 0 \\ 0
		\end{pmatrix}\}$
		\\
		\\
		.\hspace{4.7 cm} <----------- H ----------->\\
		.\hspace{4.7 cm} <--- homogenes LGS lösen um XXX zu bestimmen--->
		\\
		$\begin{pmatrix}
		1 & 1 & 1 & 1 & 0 & 0 & 0 \\
		1 & 1 & 1 & 1 & 0 & 0 & 0 \\
		1 & 0 & 1 & 0 & 1 & 0 & 1
		\end{pmatrix}
		\\ \\ \\ 
		\begin{pmatrix}
		1 & 1 & 1 & 1 & 0 & 0 & 0 \\
		0 & 0 & 1 & 1 & 1 & 1 & 0 \\
		0 & 1 & 0 & 1 & 1 & 0 & 1
		\end{pmatrix}
		\\ \\ \\
		\begin{pmatrix}
		1 & 1 & 1 & 1 & 0 & 0 & 0 \\
		0 & 1 & 0 & 1 & 1 & 0 & 1 \\
		0 & 0 & 1 & 1 & 1 & 1 & 0
		\end{pmatrix}
		\\ \\ \\
		\begin{pmatrix}
		1 & 1 & 1 & 1 & 0 & 0 & 0 \\
		0 & 1 & 0 & 1 & 1 & 0 & 1 \\
		0 & 0 & 1 & 1 & 1 & 1 & 0
		\end{pmatrix}
		\\ \\ \\
		\begin{pmatrix}
		v_1 & v_2 & v_3 & v_4 & v_5 & v_6 & v_7 \\
		1 & 1 & 1 & 1 & 0 & 0 & 0 &\vert 0\\
		1 & 1 & 1 & 1 & 0 & 0 & 0 &\vert 0\\
		1 & 0 & 1 & 0 & 1 & 0 & 1 &\vert 0
		\end{pmatrix}
		$\\
		\\
		$v_1 + v_4 + v_ + v_7 = 0$\\
		$v_1 = -v_4 - v_6 - v_7 = v_4 = v_6 = v_7$\\
		$v_4 = a$, $v_6 = c$, $v_7 = d$
		\\
		$\begin{pmatrix}
			v_1 \\ v_2 \\ v_3 \\ v_4 \\ v_5 \\ v_6 \\ v_7
		\end{pmatrix}
		=
		\begin{pmatrix}
		a + c + d \\ a + b +d \\ a + b +c \\ a \\ b \\ c \\d
		\end{pmatrix}
		$ mit $a, b, c, d \in GF(2)$\\\\
		\\\\
		Bsp. $K=\mathbb{R}$, $V=\mathbb{R}^2$ oder $\mathbb{R^3}$ (allgemein: $\mathbb{R}^n$)\\
		UVR $U_1=\{\begin{pmatrix}
		a \\ b \\ c
		\end{pmatrix} \in \mathbb{R}^3 \vert k \in \mathbb{R} \subseteq \mathbb{R}^3 \} $\\
		(gerade durch den Ursprung)\\
		$U_2=\{ k_1 \cdot \begin{pmatrix}
		a \\ b \\ c 
		\end{pmatrix} + k_2 \cdot \begin{pmatrix}
		d \\ e \\ f
		\end{pmatrix} \vert k_1, k_2 \in \mathbb{R} \}$\\
		(Ebene durch den Ursprung)
		
		
		
\end{document}
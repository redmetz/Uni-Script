\documentclass{scrartcl}
\usepackage{tabularx}
\usepackage[utf8]{inputenc}
\usepackage[T1]{fontenc}
\usepackage{lmodern}
\usepackage[ngerman]{babel}
\usepackage{amsmath}
\usepackage{amssymb}
\usepackage{stmaryrd}
\usepackage{blindtext}
\usepackage{graphicx}
\title{33. Zyklische Gruppen}
\author{Adrian Hille}
\begin{document}
\Large 33. Zyklische Gruppen\\
\\
\normalsize
Definition: Eine zyklische Gruppe falls sie von einem Element $g \in G$ erzeugt wird, d.h. $G= \{e, g, g^2, g^3, g^{-1}, g^{-2}, g^{-3}... \}$\\
\\
Beispiel: $(\mathbb{Z}, +, -, 0)$ ist eine zyklische Gruppe. 1 ist Erzeuger.\\
\\
Beispiel: $(\mathbb{Z}_3^*, \cdot, ^{-1}, 1) 2$\\
$\vert \mathbb{Z}_3^* \vert = \Phi (3) = 2 $\\
2 ist Erzeuger.\\
\\
Beispiel: ($\mathbb{Z}_5^*, \cdot, ^{-1}, 1)$\\
$\vert \mathbb{Z}_5^* \vert = \Phi (5) = 4$\\
\\
Satz: Sei p eine Primzahl. Dann ist $\mathbb{Z}_p^*$ zyklisch.\\
Zwei Gruppen sind isomorph, wenn man die eine aus der anderen durch die Umbenennung der Elemente erh\"alt.\\
Jede zyklische Gruppe ist isomorph zu $( \mathbb{Z}, +, -, 0)$\\
\\

\end{document}
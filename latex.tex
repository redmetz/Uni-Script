{\rtf1\ansi\ansicpg1252\cocoartf1404\cocoasubrtf130
{\fonttbl\f0\fmodern\fcharset0 Courier;}
{\colortbl;\red255\green255\blue255;\red0\green0\blue0;}
\paperw11900\paperh16840\margl1440\margr1440\vieww10800\viewh8400\viewkind0
\deftab720
\pard\pardeftab720\sl280\partightenfactor0

\f0\fs24 \cf2 \expnd0\expndtw0\kerning0
\outl0\strokewidth0 \strokec2 %% Erl\'e4uterungen zu den Befehlen erfolgen unter\
%% diesem Beispiel.\
\
\\documentclass\{scrartcl\}\
\
\\usepackage[utf8]\{inputenc\}\
\\usepackage[T1]\{fontenc\}\
\\usepackage\{lmodern\}\
\\usepackage[ngerman]\{babel\}\
\\usepackage\{amsmath\}\
\
\\title\{Ein Testdokument\}\
\\author\{Otto Normalverbraucher\}\
\\date\{5. Januar 2004\}\
\\begin\{document\}\
\
\\maketitle\
\\tableofcontents\
\\section\{Einleitung\}\
\
Hier kommt die Einleitung. Ihre \'dcberschrift kommt\
automatisch in das Inhaltsverzeichnis.\
\
\\subsection\{Formeln\}\
\
\\LaTeX\{\} ist auch ohne Formeln sehr n\'fctzlich und\
einfach zu verwenden. Grafiken, Tabellen,\
Querverweise aller Art, Literatur- und\
Stichwortverzeichnis sind kein Problem.\
\
Formeln sind etwas schwieriger, dennoch hier ein\
einfaches Beispiel.  Zwei von Einsteins\
ber\'fchmtesten Formeln lauten:\
\\begin\{align\}\
E &= mc^2                                  \\\\\
m &= \\frac\{m_0\}\{\\sqrt\{1-\\frac\{v^2\}\{c^2\}\}\}\
\\end\{align\}\
Aber wer keine Formeln schreibt, braucht sich\
damit auch nicht zu besch\'e4ftigen.\
\\end\{document\}\
}
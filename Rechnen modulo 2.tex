\documentclass{scrartcl}
\usepackage{tabularx}
\usepackage{float}
\usepackage[utf8]{inputenc}
\usepackage[T1]{fontenc}
\usepackage{lmodern}
\usepackage[ngerman]{babel}
\usepackage{amsmath}
\usepackage{amssymb}
\usepackage{stmaryrd}
\usepackage{blindtext}
\usepackage{graphicx}
\title{21. Rechnen modulo 2}
\author{Adrian Hille}
\begin{document}
\Large 21. Rechnen modulo 2\\
\\
\normalsize
\large allgemein: Rechnen modulo n\\
\\
\normalsize
$\mathbb{Z}_n = \{0, 1, ..., n-1\}$ (Restklassen modulo n)\\
F\"ur $a, b \in \mathbb{Z}_n$, definiere:\\
\begin{itemize}
	\item $a +_{mod~n} b := (a+b)~mod~n \in \mathbb{Z}_n$
	\item	$a \cdot_{mod~n} b := (a\cdot b)~mod~n$
	\item $a -_{mod~n} b := (a-b)~mod~n$
\end{itemize}

\large allgemein: Rechnen modulo 2\\
\normalsize
\begin{table}[H]
\begin{tabular}{l|l|l}
	$+_{mod~n} $ & 0 & 1\\
	\hline
	0 & 0 & 1\\
	\hline
	1 & 1 & 0\\
\end{tabular}
\end{table}
\begin{table}[H]
\begin{tabular}{l|l|l}
	$-_{mod~n} $ & 0 & 1\\
	\hline
	0 & 0 & 1\\
	\hline
	1 & 1 & 0\\
\end{tabular}
\end{table}
\begin{table}[H]
\begin{tabular}{l|l|l}
	$\cdot_{mod~n} $ & 0 & 1\\
	\hline
	0 & 0 & 0\\
	\hline
	1 & 0 & 1\\
\end{tabular}
\end{table}

\end{document}
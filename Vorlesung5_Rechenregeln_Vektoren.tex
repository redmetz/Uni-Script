\documentclass{scrartcl}
\usepackage[utf8]{inputenc}
\usepackage[T1]{fontenc}
\usepackage{lmodern}
\usepackage[ngerman]{babel}
\usepackage{amsmath}
\usepackage{amssymb}
\title{Skript Vorlesung 4}
\author{Lukas Jährling}
\begin{document}
	\large
	Rechenregeln für Vektoren
	\normalsize
	
	\begin{enumerate}
		\item 
		Das Nullelement in einem VR V ist eindeutig bestimmt, denn:\\
		Annahme: $0_1, 0_2$ seien Nullelemente\\
		Dann gilt: $[0_2 = 0_1 + 0_2 = 0_1]$\\
		$---v_4 Nullelemente --- v_4 Nullelemente --- nochmal nachfragen$
		
		
		\item
		Es gibt $0_k \cdot v = 0$ für jedes $v \in V$; denn:\\
		$0_k \cdot v = (0_k + 0_k) \cdot v = 0_k \cdot v + 0_k \cdot v$
		
		$NOCHMAL - MACHEN - FRAGEN$
		
		
		
		\item
		$k\cdot 0_v = 0_v$ für alle $k \in K$ \hspace{2 cm} $k\cdot 0_v = k \cdot (0_v + 0_v) = ...$
		\item
		$k_v = 0_v \Leftrightarrow k= 0_k$ oder $v = 0_v$
		
		\item
		$-v = (-1) \cdot v$ für alle $v \in V$
		
		\item
		$-k \cdot v = (-k) \cdot v$ für alle $k \in K$, $v \in V$
		
\end{enumerate}
	Definition\\
	Sei (V, $+$, $(k \vert k \in K)$) ein K-VR, U $\in$ V\\
	U heißt Untervektorraum (UVR) des K-VR V, denn:\\
	\begin{enumerate}
		\item $0_v \in U$
		\item Sei $v_1, v_2 \in U$ dann $v_1 +v_2 \in U$\\ (Abgeschlossenheit bezüglich $+$)
		\item Sei $k \in K$, $V \in U$ Denn $k \cdot V \in U$\\ (Abgeschlossenheit bezüglich Skalarmultiplikation)
	\end{enumerate}
	Bemerkung\\
	Jeder VR V hat die UVR
	\begin{enumerate}
		\item V
		\item ${0_v} (Nullraum)$
	\end{enumerate}
	$\rightarrow$ trivialer UVR
\end{document}

\documentclass{scrartcl}
\usepackage[utf8]{inputenc}
\usepackage[T1]{fontenc}
\usepackage{lmodern}
\usepackage[ngerman]{babel}
\usepackage{amsmath}
\usepackage{amssymb}
\usepackage{graphicx}
\title{Skript Vorlesung 8}
\author{Lukas Jährling}
\begin{document}
	Titel\\\\
	
	$A \in K ^{m \times n}$ \hspace{ 2cm } $rg(A)$\\
	homogenes LGS lösen $\rightarrow$ Rangberechnung\\
	Rangberechnung \begin{enumerate}
		\item 
		"direkt"
		\item
		mittels Dimensionsformel
	\end{enumerate}
	Bemerkung:\\
	Der Rang einer Matrix ändert sich nicht, wenn man elementare Zeilenumformungen anwendet.\\\\
	rg(a)=dim Col(A) = dim Row(A)\\ 
	dim Col(A) = Maximalzahl linearer unabhängiger Spaltenvektoren von A\\
	dim Row(A) = Maximalzahl linearer unabhängiger Zeilenvektoren\\\\
	Rangberechnung:\\
	$A \in K^{m \times n}$ mittels elementarer Zeilenumformungen in ZSF bringen \\\\
	---BILD (ZSF)---\\
	Der Rang von A ist die Anzahl der Zeilen, die nicht nur Null enthalten, in der ZSF.\\\\
	dim Ker(A) ist die Anzahl der freien Parameter in der Lösungsmenge von $A \cdot x = 0$.\\\\
	n=rg(A) + dim Ker(A). (Dimensionsformel)\\\\
	Satz: (Dimensionsformel)\\
	Für $A \in K ^{m \times n}$ gilt: rg(A) + dim Ker(A) = n.\\\\
	Beispiel:\\
	$A \ in \mathbb{R}^{3 \times 3}\\ \begin{pmatrix}
	1 & 1 & 1 \\ 1 & -1 & 1 \\ 1 & 0 & 1
	\end{pmatrix} \rightarrow \begin{pmatrix}
		1 & 1 & 1 \\\ 0 & -2 & 0 \\ 0 & -1 & 0
	\end{pmatrix} \rightarrow \begin{pmatrix}
		1 & 1 & 1 \\ 0 & 1 & 0 \\ 0 & 0 & 0 
	\end{pmatrix}$ \hspace{2cm} 1 freier Parameter\\
	rg(A)=2 \\
	dim Ker(A) = 3-2 = 1\\\\
	Ker(A) = $\{ \begin{pmatrix}
		-t \\ 0 \\ t
	\end{pmatrix} \vert \in \mathbb{R} \}$ \hspace{2cm}	$t = \begin{pmatrix}
		-1 \\ 0 \\ 1
	\end{pmatrix}$ Basis von Ker(A)\\\\		
	Beispiel:\\
	$A = \begin{pmatrix}
		1 & 2 \\ 3 & 4 \\ 5 & 6 \\ 7 & 8
	\end{pmatrix} \in \mathbb{R}^{4 \times 2}$ \hspace{2cm} $rg(A) <= min\{ 2, 4 \}$\\
	$A^T = \begin{pmatrix}
		1 & 3 & 5 & 7 \\ 2 & 4 & 6 & 8
	\end{pmatrix} \rightarrow \begin{pmatrix}
		1 & 3 & 5 & 7 \\ 0 & \vert \not = 0 & ... & ...
	\end{pmatrix}$\\
	$rg(A^T) = 2 \rightarrow rg(A) = rg(A^T) = 2$\\
	dim Ker(A) = n-rg(A) = 2 -2 = 0 $\rightarrow$ $Ker(A) = \{0_Z\}$\\\\
	Definition\\
	Sei $A \in K ^\{ n \times n \}$ (K Körper).\\
	a heißt invertierbar, wenn es eine Matrix $A^{-1} \in K ^{n \times n}$ mit $A \cdot A^{-1} = A ^{-1} \cdot A = E_n$ gibt.\\
	Die MAtrix A wird dann invertierbar genannt und $A^{-1}$ die zu A inverse Matrix.\\\\
	Beweis\\
	\begin{enumerate}
		\item
		$A ^{-1}$ ist eindeutig bestimmt (falls $A^{-1}$ existiert), denn:\\
		Sei $A \cdot A^{-1}_1 = A^{-1}_2 \cdot A = E_n$, \\
		$A \cdot A^{-1}_2 = A^{-1}_2 \cdot A = E_n$\\
		Dann: $A \cdot A^{-1}_1 = A \cdot A^{-1}_2 \vert A^{-1}_1$\\
		$(A^{-1}_1 \cdot A) \cdot A ^{-1}_1 = (A^{-1}_1 \cdot A) \cdot A^{-1}_2 \rightarrow E_n \cdot A ^{-1}_1 = E_n \cdot A^{-1}_2 \rightarrow A^{-1}_1 \cdot A^{-1}_2$
		\item
		$(A \cdot B)^{-1} = B^{-1} \cdot A ^{-1}$ (Reihenfolge beachten)
	\end{enumerate}
	Bemerkung\\
	Zur Berechnung von $A^-1$ sind n LGS zu lösen (simultan);\\
	damit kann man gleichzeitig entscheiden, ob $A^{-1}$ existiert\\
	\begin{enumerate}
		\item
		Notiere $(A \vert E_n)$
		\item
		Überführung im ZSF \\
		---BILD---\\
		Enthält diese Matrix eine Nullzeile, dann existiert $A^{-1}$ nicht.\\
		---Bild---\\
		Enthält diese Matrix keine Nullzeile, dann existiert $A^{-1}$
		\item
		Überführt in reduzierte ZSF $\begin{pmatrix}
			1 & 0 & ... & 0 &\vert & \\
			& 1 & ... & ... & \vert & A^{-1}\\
			0 &  &  & 1 & \vert & 
		\end{pmatrix}$
	\end{enumerate}
	Beispiel\\
	$
	A = \begin{pmatrix}
		1 & 2 & 3 \\ 2 & 5 & 3 \\ 1 & 0 & 8
	\end{pmatrix}$\\
	
	$\begin{pmatrix}
		1 & 2 & 3 & \vert & 1 & 0 & 0 \\
		2 & 5 & 3 & \vert & 0 & 1 & 0 \\
		1 & 0 & 8 & \vert & 0 & 0 & 1
	\end{pmatrix} \rightarrow ... \rightarrow \begin{pmatrix}
		1 & 2 & 3 & \Vert & 1 & 0 & 0 \\ 0 & 1 & -3 & \Vert & -2 & 1 & 0 \\ 0 & 0 & -1 & \vert & -5 & 2 & 1  
	\end{pmatrix}$
	\\ ABSCHREIBEN \\\\
	Bemerkung\\
	$\begin{pmatrix}
		d_1 & 0 \\ 0 & d_n
	\end{pmatrix} ^{-1} = \begin{pmatrix}
		d_1^{-1} & 0 \\ 0 & d_n^{-1}
	\end{pmatrix}$, falls $d_1 \not = 0 , ... , d_n \not = 0$\\\\
	Bemerkung\\
	$A = \begin{pmatrix}
		a & b \\ c & d
	\end{pmatrix} \rightarrow A^{-1} = 1 / {a \cdot d - b \cdot c} \cdot \begin{pmatrix}
		d & -b  \\ -c & a 
	\end{pmatrix}$, falls $a \cdot d - b \cdot c \not = 0$,\\ denn:\\
	$\begin{pmatrix}
		a & b \\ c & d
	\end{pmatrix} \cdot 1 / {a \cdot d - b \cdot c} \cdot \begin{pmatrix}
		d & -b \\ -c & a
	\end{pmatrix} = 1 / {a \cdot d - b \cdot c} \cdot \begin{pmatrix}
	a \cdot d - b \cdot c & 0 \\ 0 & a \cdot d - b  \cdot c
	\end{pmatrix} = \begin{pmatrix}
		1 & 0 \\ 0 & 1
	\end{pmatrix}$\\\\
	Satz\\
	Sei $A \in K ^{n \times n}$\\
	$A ^{-1}$ existiert $\rightleftarrows rg(A) = n$\\
	$\rightarrow$ dim Ker(A) = n - n = 0\\
	$\rightarrow$ Ker (A) = $0_{K^n}$ Nullvektor\\
	$\rightarrow$ Ax = 0 hat nur die triviale Lösung (Nullvektor)\\
	\newpage
	5. Lineare Abbildungen\\\\
	Definition\\
	Sei V, W K-VR. Eine Abbildung $f: V \rightarrow W$ heißt linear (oder Homomorphismus), wenn:\\
	\begin{enumerate}
		\item
		$f(a+_V b) = f(a) +_W f(b)$ für $a,b \in V$
		\item
		$f(k \cdot a) = k \cdot f(a)$ für $a \in V , k \in K$
	\end{enumerate}
	Bemerkung\\
	$f(k_1 \cdot a_1 + ... + k_n \cdot a_n) = k_1 \cdot f (a_1) + ... + k_n \cdot f(a_n)$\\
	für $k_1 , ... , k_n \in K,$ \hspace{1cm} $a_1 , ... , a_n \in V $\\\\
	Bemerkung\\
	Lineare Abbildungen sind strukturverträglich\\\\
	Bemerkung\\
	$f(0_V) = 0_W$, \\denn: $f(0) = f(a - a) = f(a + (-1) \cdot a) = f(a)+ (-1) \cdot f(a) = f(a) - f(a) = 0_W$
\end{document}
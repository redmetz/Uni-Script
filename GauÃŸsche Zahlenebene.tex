\documentclass{scrartcl}
\usepackage[utf8]{inputenc}
\usepackage[T1]{fontenc}
\usepackage{lmodern}
\usepackage[ngerman]{babel}
\usepackage{amsmath}
\usepackage{amssymb}
\usepackage{graphicx}
\title{Lineare Algebra - Gaußsche Zahlenebene}
\author{Tobias Raaf}
\begin{document}
	\Large Die Gaußsche Zahlenebene \\\\\\
	\normalsize
\includegraphics{GaussEbene.png}\\\\\\
Bemerkungen: z ist eindeutig durch (a,b) bestimmt.\\
\hspace*{68pt} z ist eindeutig durch (r,$\varphi$) bestimmt.\\
Anhand des Satzes des Pythagoras lässt sich aus der arithmetischen Form (a+bi), die trigonometrische Form berechnen:\\
r=$\sqrt{a^2+b^2}$\\
$\Rightarrow$ $\sin(\varphi)$=$\dfrac{b}{r}$, $\cos(\varphi)$=$\dfrac{a}{r}$, \\
$\Rightarrow$ z = a+bi = $r\cos(\varphi)+r\sin(\varphi)i$ = $r(\cos(\varphi)+\sin(\varphi)i$)\\\Large
\textbf{= $re^{i\varphi}$} $\Rightarrow$ Exponentielle/Eulersche Darstellung\\\normalsize
Bemerkung: r=$\sqrt{a^2+b^2}$=|z| nennt man \textbf{Betrag} von z=a+bi\\
\hspace*{56pt} $\varphi$ nennt man \textbf{Argument} von z=a+bi\\
Bemerkung zum Argument: $\varphi$ mit 0$\leq$ $\varphi$ < $2\pi$ nennt man \textbf{Hauptargument}\\
\\\\
\textbf{Multiplikation/Division in Eulerscher Darstellung}:\\
$z_1=r_1*e^{i\varphi_1},z_2=r_2*e^{i\varphi_2}$\\
\textbf{Multiplikation}: $z_1*z_2$ = $r_1*e^{i\varphi_1}*r_2*e^{i\varphi_2}$=$(r_1*r_2)e^{i(\varphi_1*\varphi_2)}$
\textbf{Division}= $z_2\neq0$\\
$\dfrac{z_1}{z_2}=\dfrac{r_1*e^{i\varphi_1}}{r_2*e^{i\varphi_2}}= \dfrac{r_1}{r_2}*e^{i(\varphi_1-\varphi_2)}$ 

\end{document}
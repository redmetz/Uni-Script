\documentclass{scrartcl}\usepackage[utf8]{inputenc}
\usepackage[T1]{fontenc}
\usepackage{lmodern}
\usepackage[ngerman]{babel}
\usepackage{amsmath}
\usepackage{amssymb}
\usepackage{stmaryrd}
\usepackage{blindtext}
\title{2. Mengenangaben durch Aussonderung}
\author{Adrian Hille}
\begin{document}
\Large 2. Mengenangaben durch Aussonderung
\\
\\
\normalsize
$\{x \mid Bedingung (x) \}$:  Die Mengen aller x, die die gegebene Bedingung erf\"ullen.\\
z.B.: alle geraden x
$\{x \mid \exists y \quad y\cdot 2=x) \}$
\\
  \begin{itemize}
    	\item $A \cap B$: Schnitt von A und B.
    	\item $A \cup B$: Vereinigung von A und B.
  	\item $A \setminus B$: Differenz von A und B.
    \end{itemize}
\end{document}